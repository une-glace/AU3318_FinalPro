\documentclass{article}
\usepackage{ctex}     % 导入 ctex 宏包
\usepackage{amsmath}  % 提供数学公式排版功能
\usepackage{amssymb}  % 提供额外的数学符号
\usepackage{graphicx} % 插入图片
\usepackage{float}    % 图表位置
\usepackage{geometry} % 控制页面布局
\usepackage{subcaption}
\usepackage{listings}
\usepackage{xcolor}
\geometry{a4paper, margin=1in} % 设置纸张大小和边距
\lstset{
    basicstyle=\ttfamily\footnotesize,
    keywordstyle=\color{blue},
    commentstyle=\color{gray},
    stringstyle=\color{red},
    numbers=left,
    numberstyle=\tiny\color{gray},
    stepnumber=1,
    numbersep=5pt,
    backgroundcolor=\color{white},
    showspaces=false,
    showstringspaces=false,
    showtabs=false,
    frame=single,
    tabsize=2,
    captionpos=b,
    breaklines=true,
    breakatwhitespace=false,
    escapeinside={\%*}{*)}
}

% 设置中文字体为楷体
\setCJKmainfont{KaiTi} % 设置主字体为楷体
\setCJKsansfont{KaiTi} % 设置无衬线字体为楷体
\setCJKmonofont{KaiTi} % 设置等宽字体为楷体

\title{AU3318 大作业报告}
\author{李青雅 523030910004 电院2301}
\date{\today} % 自动插入今天的日期

\begin{document}

\maketitle
\newpage
\tableofcontents % 生成目录
\newpage

\section{实验目的}
本实验旨在利用数字信号处理技术,对发电机定子槽楔的敲击振动信号进行分析。通过对不同紧固压力下(400N - 4000N)采集的声学/振动信号进行时域切分、频域变换及特征提取,构建分类模型以识别槽楔的松紧状态,并探究松紧度与信号频率特征之间的定量关系。

\section{数据预处理与时域分析}
\subsection{数据读取}
实验数据为一系列 TXT 文本文件,记录了不同压力下敲击测试的连续采样点。利用 Python 的 \texttt{numpy} 库读取数据,将文本序列转换为内存中的数值数组。

\begin{figure}[H]
    \centering
    \includegraphics[width=1\textwidth]{../waveform_preview.png}
    \caption{原始采集数据的时域波形示例(acquisitionData-400.txt)}
    \label{fig:raw_waveform}
\end{figure}

\subsection{时域信号切分}
原始数据包含连续多次敲击的信号。为了进行特征分析,首先需要将单次敲击信号从长录音中分离出来。
本实验采用\textbf{动态阈值法}进行切分:
\begin{enumerate}
    \item 移除直流分量:$x[n] = x[n] - \text{mean}(x)$
    \item 设定阈值:取信号最大幅值的 15\% 作为触发阈值。
    \item 峰值检测与截取:当信号幅度超过阈值且距离上一次检测超过一定间隔(2000点)时,判定为一次新的敲击。
\end{enumerate}

\begin{figure}[H]
    \centering
    \includegraphics[width=1\textwidth]{../waveform_with_peaks.png}
    \caption{时域波形及敲击事件检测结果(红叉标记为检测到的敲击点)}
    \label{fig:waveform}
\end{figure}

\subsection{时域信号长度的选择}
在切分过程中,我们选定每个样本的长度为 \textbf{4096} 个采样点(约 40ms,假设采样率 100kHz)。

\textbf{选定合适的时域信号长度的原因}
\begin{itemize}
    \item \textbf{频率分辨率}:FFT 的频率分辨率 $\Delta f = F_s / N$。选择 4096 点($2^{12}$)不仅利用了 FFT 算法的高效性,更重要的是提供了足够高的频率分辨率,以便区分不同松紧度下的细微频率差异。
    \item \textbf{信噪比控制}:敲击信号是瞬态阻尼振荡。如果截取过短,会丢失信号尾部的有效信息;如果截取过长,则会引入大量的背景静音噪声,降低信噪比。4096 点恰好能完整覆盖一次敲击的起振与衰减过程。
    \item \textbf{样本一致性}:机器学习分类器要求输入特征维度一致,因此必须统一所有样本的长度。
\end{itemize}

\section{频域分析}
\subsection{频域变换}
对截取出的 4096 点时域信号 $x[n]$ 进行快速傅里叶变换(FFT),将其转换为频域信号 $X[k]$。
在变换前,先对信号加 \textbf{Hanning 窗},以减小频谱泄漏(Spectral Leakage)。
Hanning 窗的数学表达式为:
\begin{equation}
w(n) = 0.5 \left( 1 - \cos \left( \frac{2\pi n}{N-1} \right) \right), \quad 0 \le n \le N-1
\end{equation}
其中,$N$ 为窗口长度(本实验中 $N=4096$)。

\textbf{处理时域和频域关系的区别:}
\begin{itemize}
    \item \textbf{步骤3(时域定长)}是\textbf{预处理}阶段,目的是准备合格的“原材料”。它决定了后续分析的时间窗口和基本分辨率。
    \item \textbf{步骤4(频域变换)}是\textbf{特征空间转换}阶段,目的是将信号从“时间-幅度”视角切换到“频率-能量”视角。
    \item \textbf{关系}:时域是信号的真实物理表现(随时间变化),频域是信号的数学统计特性(包含哪些频率成分)。对于槽楔松紧度检测,时域波形受敲击力度影响较大(力大则幅值大),不稳定;而频域特征主要由物体固有的物理属性(刚度、质量)决定,更能反映“松紧”这一本质特征。
\end{itemize}

\subsection{频谱特征对比}
通过对比不同压力下的频谱图(图 \ref{fig:spectrum}),可以发现:
\begin{itemize}
    \item \textbf{低压力(松)}:主峰频率较低,且频谱分布较散乱,包含较多低频杂波。
    \item \textbf{高压力(紧)}:主峰频率明显向高频移动,且能量更加集中。
\end{itemize}

\begin{figure}[H]
    \centering
    \includegraphics[width=1\textwidth]{../spectrum_comparison.png}
    \caption{不同压力下的归一化频谱对比}
    \label{fig:spectrum}
\end{figure}

\section{特征提取与分类}
为了量化信号特征,本实验提取了多域特征进行融合分析:
\begin{itemize}
    \item \textbf{频域特征}:主频(Peak Frequency)、频谱质心(Spectral Centroid)、带宽(Bandwidth)、谱偏度(Spectral Skewness)、能量(Energy)。
    \item \textbf{时域特征}:峰度(Kurtosis,反映波形尖锐度)、偏度(Skewness,反映波形对称性)、均方根值(RMS,反映有效强度)。
\end{itemize}

\subsection{分类模型与精度}
构建了包含 382 个样本的数据集,并划分 70\% 训练集和 30\% 测试集。
使用 \textbf{随机森林(Random Forest)} 分类器进行训练,最终在测试集上达到了 \textbf{86.96\%} 的准确率。

\begin{table}[H]
\centering
\begin{tabular}{|c|c|c|c|}
\hline
\textbf{压力类别} & \textbf{Precision} & \textbf{Recall} & \textbf{F1-Score} \\ \hline
400 (很松) & 1.00 & 1.00 & 1.00 \\ \hline
1200 (较松) & 1.00 & 1.00 & 1.00 \\ \hline
2000 (中等) & 0.81 & 0.93 & 0.87 \\ \hline
4000 (很紧) & 0.67 & 0.67 & 0.67 \\ \hline
\end{tabular}
\caption{部分压力类别的分类性能表}
\end{table}

实验结果表明,模型对低压力(松动)状态的识别非常精准(F1=1.00),能有效检出故障;而在高压力区,由于物理特性趋于饱和,区分难度较大,但在去除低频干扰后,整体准确率有了显著提升。

\begin{figure}[H]
    \centering
    \includegraphics[width=0.8\textwidth]{../confusion_matrix.png}
    \caption{分类结果混淆矩阵}
    \label{fig:confusion}
\end{figure}

\section{松紧度与频率的关系}

通过统计所有样本的平均特征,绘制了压力值与频率特征的关系曲线(图 \ref{fig:trend})。

\textbf{特别说明}:在初步分析中,我们发现高压力(4000N)样本在 1000Hz 附近存在异常强的低频分量(可能是台架共振或刚体模态),这掩盖了反映接触刚度的高频特征,导致频率与压力关系出现反常。因此,在特征提取时我们\textbf{滤除了 1500Hz 以下的低频成分}。

修正后的结果显示:
\begin{itemize}
    \item \textbf{趋势}:随着压力值(松紧度)的增加,信号的特征频率(尤其是频谱质心 Spectral Centroid)呈现非常显著的\textbf{上升趋势}(从约 6000Hz 上升至 9500Hz)。
    \item \textbf{主频的波动}:图中蓝线(主频 Peak Frequency)在某些压力点(如 1600N)出现大幅波动。这是由于槽楔结构存在多个相近的固有频率(多模态),当敲击位置或力度微小变化时,最大峰值可能在两个模态之间跳变(Mode Jumping)。相比之下,\textbf{频谱质心(Spectral Centroid)}作为频谱能量的重心,综合了所有频率成分,因此表现出极好的稳定性和单调性。
    \item \textbf{物理机制}:根据振动理论,结构的固有频率 $f \propto \sqrt{k/m}$。随着槽楔压紧力增大,接触刚度 $k$ 增大,导致固有频率升高。
    \item \textbf{结论}:频率特征(特别是质心频率)与松紧度存在显著的正相关性,验证了利用声学/振动信号检测槽楔松紧度的可行性。
\end{itemize}

\begin{figure}[H]
    \centering
    \includegraphics[width=1\textwidth]{../final_result_trend.png}
    \caption{松紧度(压力)与频率特征的变化关系}
    \label{fig:trend}
\end{figure}

\end{document}